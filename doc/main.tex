\documentclass{article}

\usepackage[left=2cm,right=2cm,top=2cm,bottom=2cm]{geometry}

\begin{document}

\newcommand{\NL}{\mathbf{N_L}}
\newcommand{\NR}{\mathbf{N_R}}
\newcommand{\ML}{\mathbf{M_L}}
\newcommand{\MR}{\mathbf{M_R}}
\newcommand{\AL}{\mathbf{A_L}}
\newcommand{\AR}{\mathbf{A_R}}
\newcommand{\Aroe}{\lvert\mathbf{A}\rvert}
\newcommand{\rotmat}{\mathbf{T}}
\newcommand{\flux}{\mathbf{F}}
\newcommand{\cons}{\mathbf{U}}
\newcommand{\invrotmat}{\mathbf{T}^{-1}}

\begin{equation}
    \mathcal{H} \cdot \mathbf{n}
    =
    \invrotmat\flux(\rotmat\cons)
\end{equation}

\begin{equation}
    \invrotmat\flux(\rotmat\cons^{n+1})
    =
    \invrotmat\flux(\rotmat\cons^{n})
    +
    \invrotmat\frac{\partial\flux(\rotmat\cons)}{\partial(\rotmat\cons)}\Delta(\rotmat\cons)
\end{equation}

\begin{equation}
    \invrotmat\frac{\partial\flux(\rotmat\cons)}{\partial(\rotmat\cons)}\Delta(\rotmat\cons)
    =
    \invrotmat
    \left[
        \frac{\partial\flux(\rotmat\cons_L)}{\partial(\rotmat\cons_L)}\Delta(\rotmat\cons_L)
        +
        \frac{\partial\flux(\rotmat\cons_R)}{\partial(\rotmat\cons_R)}\Delta(\rotmat\cons_R)
    \right]
\end{equation}

\begin{equation}
    \invrotmat\frac{\partial\flux(\rotmat\cons)}{\partial(\rotmat\cons)}\Delta(\rotmat\cons)
    =
    \invrotmat
    \left[
        \frac{\AL + \Aroe}{2}\Delta(\rotmat\cons_L)
        +
        \frac{\AR - \Aroe}{2}\Delta(\rotmat\cons_R)
    \right]
\end{equation}

\begin{equation}
    \invrotmat\frac{\partial\flux(\rotmat\cons)}{\partial(\rotmat\cons)}\Delta(\rotmat\cons)
    =
    \invrotmat
    \left[
        \frac{\AL + \Aroe}{2}\rotmat\Delta\cons_L
        +
        \frac{\AR - \Aroe}{2}\rotmat\Delta\cons_R
    \right]
\end{equation}

\begin{equation}
    \invrotmat\frac{\partial\flux(\rotmat\cons)}{\partial(\rotmat\cons)}\Delta(\rotmat\cons)
    =
    \NL\Delta\cons_L
    -
    \NR\Delta\cons_R
\end{equation}

where, 

\begin{equation}
    \NL
    =
    (\invrotmat\frac{\AL + \Aroe}{2}\rotmat)\Delta\cons_L
\end{equation}

\begin{equation}
    \NR
    =
    -(\invrotmat\frac{\AR - \Aroe}{2}\rotmat)\Delta\cons_R
\end{equation}

Consider first order forward Euler temporal discretization,

\begin{equation}
    V\frac{\cons^{n+1} - \cons^{n}}{\Delta t}
    =
    -\sum\invrotmat\flux(\rotmat\cons)S
\end{equation}

\begin{equation}
    V\frac{\Delta\cons}{\Delta t}
    =
    -\sum\invrotmat\flux(\rotmat\cons)S
\end{equation}

A cell can be either on the left or on the right of a face. If a cell is on the left of a face,

\begin{equation}
    V\frac{\Delta\cons_L}{\Delta t}
    =
    -\NL\Delta\cons_LS
    +
    \NR\Delta\cons_RS
\end{equation}

Define $\ML_K = \NL_K S$.

\begin{equation}
    \left(
        \frac{V}{\Delta t} + \ML
    \right)
    \Delta\cons_L
    =
    \MR\Delta\cons_R
\end{equation}

If a cell is on the right of a face, flux through a face is opposite to that for the left cell, that is, $-\invrotmat\flux(\rotmat\cons) \rightarrow \invrotmat\flux(\rotmat\cons)$.

\begin{equation}
    V\frac{\Delta\cons_R}{\Delta t}
    =
    \NL\Delta\cons_LS
    -
    \NR\Delta\cons_RS
\end{equation}

\begin{equation}
    \left(
        \frac{V}{\Delta t} + \MR
    \right)
    \Delta\cons_R
    =
    \ML\Delta\cons_L
\end{equation}

Now consider second order three-time level backward Euler discretization.

\begin{equation}
    V\frac{3q^{n+1}-4q^n+q^{n-1}}{2\Delta t}
    =
    -\sum\invrotmat\flux(\rotmat\cons)S
\end{equation}





\end{document}
