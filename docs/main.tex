\documentclass{article}

\usepackage[left=2cm,right=2cm,top=2cm,bottom=2cm]{geometry}
\usepackage{amsmath,amssymb}
\usepackage{booktabs}

\begin{document}

\newcommand{\Mi}{\mathbf{M}_i}
\newcommand{\Mj}{\mathbf{M}_j}
\newcommand{\NL}{\mathbf{N}_L}
\newcommand{\NR}{\mathbf{N}_R}
\newcommand{\ML}{\mathbf{M}_L}
\newcommand{\MR}{\mathbf{M}_R}
\newcommand{\AL}{\mathbf{A}_L}
\newcommand{\AR}{\mathbf{A}_R}
\newcommand{\Aroe}{\lvert\mathbf{A}\rvert}
\newcommand{\rotmat}{\mathbf{T}}
\newcommand{\flux}{\mathbf{F}}
\newcommand{\cons}{\mathbf{U}}
\newcommand{\invrotmat}{\mathbf{T}^{-1}}
\newcommand{\R}{\mathbf{R}}
\newcommand{\T}{\mathbf{T}}

\begin{equation}
    \sum\mathcal{H} \cdot \mathbf{n}S
    =
    \sum\invrotmat\flux(\rotmat\cons)S
\end{equation}

\begin{equation}
    \invrotmat\flux(\rotmat\cons^{n+1})S
    =
    \invrotmat\flux(\rotmat\cons^{n})S
    +
    \invrotmat\frac{\partial\flux(\rotmat\cons^n)}{\partial(\rotmat\cons^n)}\Delta(\rotmat\cons^n)S
\end{equation}

For simplicity, drop $()^n$. 

\begin{equation}
    \invrotmat\frac{\partial\flux(\rotmat\cons)}{\partial(\rotmat\cons)}\Delta(\rotmat\cons)
    =
    \invrotmat
    \left[
        \frac{\partial\flux(\rotmat\cons_L)}{\partial(\rotmat\cons_L)}\Delta(\rotmat\cons_L)
        +
        \frac{\partial\flux(\rotmat\cons_R)}{\partial(\rotmat\cons_R)}\Delta(\rotmat\cons_R)
    \right]
\end{equation}

\begin{equation}
    \invrotmat\frac{\partial\flux(\rotmat\cons)}{\partial(\rotmat\cons)}\Delta(\rotmat\cons)
    =
    \invrotmat
    \left[
        \frac{\AL + \Aroe}{2}\Delta(\rotmat\cons_L)
        +
        \frac{\AR - \Aroe}{2}\Delta(\rotmat\cons_R)
    \right]
\end{equation}

\begin{equation}
    \invrotmat\frac{\partial\flux(\rotmat\cons)}{\partial(\rotmat\cons)}\Delta(\rotmat\cons)
    =
    \invrotmat
    \left[
        \frac{\AL + \Aroe}{2}\rotmat\Delta\cons_L
        +
        \frac{\AR - \Aroe}{2}\rotmat\Delta\cons_R
    \right]
\end{equation}

\begin{equation}
    \invrotmat\frac{\partial\flux(\rotmat\cons)}{\partial(\rotmat\cons)}\Delta(\rotmat\cons)S
    =
    \ML\Delta\cons_L
    -
    \MR\Delta\cons_R
\end{equation}

\begin{equation}
    \ML
    =
    \left(
        \invrotmat\frac{\AL + \Aroe}{2}\rotmat
    \right)
    \Delta\cons_LS
\end{equation}

\begin{equation}
    \MR
    =
    -\left(
        \invrotmat\frac{\AR - \Aroe}{2}\rotmat
    \right)
    \Delta\cons_RS
\end{equation}

Set $()$ to $()^n$ back again.

\begin{equation}
    \sum\invrotmat\flux(\rotmat\cons^{n+1})S
    =
    \sum\invrotmat\flux(\rotmat\cons^{n})S
    +
    \sum\ML\Delta\cons^{n}_L
    -
    \sum\MR\Delta\cons^{n}_R
\end{equation}

\begin{equation}
    \sum\R(\cons^{n+1})
    =
    \sum\invrotmat\flux(\rotmat\cons^{n})S
    +
    \sum\ML\Delta\cons^{n}_L
    -
    \sum\MR\Delta\cons^{n}_R
\end{equation}

A cell can be a left or right cell. 

\begin{equation}
    \sum\R(\cons^{n+1}_i)
    =
    \sum\invrotmat\flux(\rotmat\cons^{n}_i)S
    +
    \sum\Mi\Delta\cons^{n}_i
    +
    \sum\Mj\Delta\cons^{n}_j
\end{equation}

Consider a generic temporal discretization:

\begin{equation}
    \alpha V\frac{\Delta\cons}{\Delta t}
    -
    \T
    +
    \sum\R(\cons^{n+1}_i)
    =
    0
\end{equation}

where, $\alpha$ is a coefficient depending on temporal discretization and $\T$ is higher order terms of the temporal discretization. Rearranging the equation:

\begin{equation}
    \alpha V\frac{\Delta\cons}{\Delta t}
    =
    -\sum\R(\cons^{n+1}_i)
    +
    \T
\end{equation}

\begin{equation}
    \left(
        \alpha\frac{V}{\Delta t} + \sum\Mi
    \right)
    \Delta\cons_i
    =
    -\sum\R(\cons_i)
    +
    \sum\Mj\Delta\cons_j
    +
    \T
\end{equation}

where, for each cell-face,

\begin{equation}
    \Mi =
    \begin{cases}
        \ML & \text{if $i$ is on the left}\\
        -\MR & \text{if $i$ is on the right}
    \end{cases}
\end{equation}

\begin{equation}
    \Mj =
    \begin{cases}
        -\MR & \text{if $i$ is on the left}\\
        \ML & \text{if $i$ is on the right}
    \end{cases}
\end{equation}

For example, consider three-time level backward Euler discretization.

\begin{equation}
    V\frac{3\cons^{n+1}-4\cons^n+\cons^{n-1}}{2\Delta t}
    =
    -\sum\R(\cons^{n+1}_i)
\end{equation}

Separate the first and higher order terms of temporal discretization.

\begin{equation}
    V\frac{3\cons^{n+1}-4\cons^n+\cons^{n-1}}{2\Delta t}
    &=
    \frac{1}{2}\frac{V}{\Delta t}
    [
    3\Delta\cons
    -
    (\cons^{n} - \cons^{n-1})
    ]
\end{equation}

\begin{equation}
    \frac{3}{2}\frac{V}{\Delta t}\Delta\cons
    =
    -\sum\R(\cons^{n+1}_i) + \frac{1}{2}\frac{V}{\Delta t}(\cons^{n} - \cons^{n-1})
\end{equation}

\begin{equation}
    \left(
        \frac{3}{2}\frac{V}{\Delta t} + \Mi
    \right)
    \Delta\cons_i
    =
    -\sum\R(\cons^{n}_i) + \sum\Mj\Delta\cons_j + \frac{1}{2}\frac{V}{\Delta t}(\cons^{n} - \cons^{n-1})
\end{equation}

\begin{table}
    \centering
    \begin{tabular}{lll}
        Implicit Scheme & $\alpha$ & $\T$\\
        \midrule
        Euler & 1 & $[0]^T$\\
        Three-time level & $\frac{3}{2}$ & $\frac{1}{2}\frac{V}{\Delta t}(\cons^{n} - \cons^{n-1})$
    \end{tabular}
\end{table}




\end{document}
